\documentclass[a4j,10pt]{jsarticle}
\usepackage{mylatex}
\usepackage{listings,jlisting}
\usepackage[dvipdfmx]{graphicx}
\usepackage{subcaption}
\usepackage{here}
\graphicspath{ {figures/} }

\title{プロジェクト実習P PHP演習}
\author{C0116023 飯島貴政}
\date{2012年4月27日(金)}

\begin{document}
\maketitle


\section{目的}

今回の課題の目的を述べる。

\lstset{
    language = PHP,
    %枠外に行った時の自動改行
    breaklines = true,
    %自動改行後のインデント量(デフォルトでは20[pt])
    breakindent = 10pt,
    %標準の書体
    basicstyle = \ttfamily\scriptsize,
    %関数名等の色の設定
    classoffset = 0,
    %枠 “t”は上に線を記載, “T”は上に二重線を記載
    %他オプション:leftline、topline、bottomline、lines、single、shadowbox
    frame = TBrl,
    %frameまでの間隔(行番号とプログラムの間)
    framesep = 5pt,
    %行番号の位置
    numbers = left,
    %行番号の間隔
    stepnumber = 1,
    %行番号の書体
    numberstyle = \tiny,
    %タブの大きさ
    tabsize = 4,
    %キャプションの場所(“tb”ならば上下両方に記載)
    captionpos = t
}
%=================================================================================

\section{課題1}

\subsection{問題}

文字列「Hello World」をブラウザ上に表示するプログラムphp1.phpを作成せよ.
以降の課題はHTMLにPHPプログラムを埋め込むことによって構成すること.HTMLは5に準拠すること.


\subsection{ソースコード}

\lstinputlisting[caption=PHP1.php,label=pg:k1-1s]{src/php1.php}

\pgref{pg:k1-1s}のソース8行目でHTML5に準拠させるためhtml lang及び,5行目で文字エンコードを設定している.
ソース8行目でPHPのechoにより {\tt Hello World }という文字列を出力している。

\subsection{実行結果}

\begin{figure}[H]
  \centering
  \fbox{
  \includegraphics{1.png}
  }
  \caption{リスト1の実行結果}
  \label{fig:boat1}
\end{figure}
%=================================================================================

\section{課題2}
\subsection{問題}

\begin{verbatim}
配列にある住所録を表示する関数print_table($addresses)を含むプログラムphp2.php作成せよ.
このとき¥addressesは住所録を保持する配列であるとする.これは,1名分のデータを保持する連想配列を,要素として人数分含む二重配列構造を持つ.
1名分のデータを保持する連想配列はインデックスとしてname/address/phone/emailを持つものとする.
\end{verbatim}
\subsection{ソースコード}

\lstinputlisting[caption=PHP2.php,label=pg:k2-1s]{src/php2.php}

\pgref{pg:k2-1s}のソース11行目から14行目でサンプルの2人分のデータを\$addressesという変数に連想配列で格納した.
18行目でtableタグを宣言し,表を描画する準備をする.24行目でデータの見出しを表示している.29行目でforeachを繰り返しデータの内容をキーごとに出力している.

\subsection{実行結果}

\begin{figure}[H]
  \centering
  \fbox{
  \includegraphics[width=10cm]{2.png}
  }
  \caption{リスト2の実行結果}
  \label{fig:boat1}
\end{figure}
%=================================================================================

\section{課題3}

\subsection{問題}

\begin{verbatim}
POSTメソッドを使用して,住所録に項目を追加するプログラムphp3.phpを作成せよ.
住所録のデータは保存されず,読み込みのたびに初期状態に戻ってよい.
\end{verbatim}
\subsection{ソースコード}

\lstinputlisting[caption=PHP3.php,label=pg:k2-1s]{src/php3.php}
\begin{verbatim}
\pgref{pg:k2-1s}のソース13行目から17行目でサンプルの2人分のデータを$addressesという変数に連想配列で格納した.
19行目から送信ボタンが押された際の動作を定義している。$getPostという変数にフォームから送られたデータを$_POSTで取得し、連想配列に格納している。
23行目のarray_pushの第1引数をデータが追加される連想配列$addresses、第2引数を追加するデータである$getPostにすることで、住所録に新しいデータを追加することができる。
57行目から実際にwebページに表示するフォームを設定している。メソッドはPOSTである。
各フォームのnameタグはPHP側の連想配列$addressesに対応するようにした。

\end{verbatim}
\subsection{実行結果}

\begin{figure}[H]
  \centering
  \fbox{
  \includegraphics[width=10cm]{3.png}
  }
  \caption{リスト3の実行結果[1]}
\label{fig:fig3}
\end{figure}

\begin{figure}[H]
  \centering
  \fbox{
  \includegraphics[width=10cm]{4.png}
  }
  \caption{リスト3の実行結果[2]}
\label{fig:fig4}
\end{figure}

\begin{figure}[H]
  \centering
  \fbox{
  \includegraphics[width=10cm]{5.png}
  }
  \caption{リスト3の実行結果[3]}
\label{fig:fig5}
\end{figure}

\figref{fig:fig3}はphp3.phpページを開いた時の初期画面である。
\figref{fig:fig4}は入力フォームに追加したいデータを入力している。
\figref{fig:fig5}は送信ボタンを押した後、表に新規データが追加されていることが確認できた。
%=================================================================================

\section{課題4}

\subsection{問題}

\begin{verbatim}
php3.phpにおいて、データが追加された際に住所録の内容をJSON形式を用いて表示するよ
うにせよ。
(JSON形式による配列の表示は、デバッグの手段としても有効である)

\end{verbatim}
\subsection{ソースコード}

\lstinputlisting[caption=PHP4.php,label=pg:k2-1s]{src/php3.php}
\begin{verbatim}
\pgref{pg:k2-1s}のソース13行目から17行目でサンプルの2人分のデータを$addressesという変数に連想配列で格納した.
19行目から送信ボタンが押された際の動作を定義している。$getPostという変数にフォームから送られたデータを$_POSTで取得し、連想配列に格納している。
23行目のarray_pushの第1引数をデータが追加される連想配列$addresses、第2引数を追加するデータである$getPostにすることで、住所録に新しいデータを追加することができる。
57行目から実際にwebページに表示するフォームを設定している。メソッドはPOSTである。
各フォームのnameタグはPHP側の連想配列$addressesに対応するようにした。

\end{verbatim}
\subsection{実行結果}

\begin{figure}[H]
  \centering
  \fbox{
  \includegraphics[width=10cm]{3.png}
  }
  \caption{リスト3の実行結果[1]}
\label{fig:fig3}
\end{figure}

\begin{figure}[H]
  \centering
  \fbox{
  \includegraphics[width=10cm]{4.png}
  }
  \caption{リスト3の実行結果[2]}
\label{fig:fig4}
\end{figure}

\begin{figure}[H]
  \centering
  \fbox{
  \includegraphics[width=10cm]{5.png}
  }
  \caption{リスト3の実行結果[3]}
\label{fig:fig5}
\end{figure}

\figref{fig:fig3}はphp3.phpページを開いた時の初期画面である。
\figref{fig:fig4}は入力フォームに追加したいデータを入力している。
\figref{fig:fig5}は送信ボタンを押した後、表に新規データが追加されていることが確認できた。
%=================================================================================




\end{document}
