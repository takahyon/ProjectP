\documentclass[a4j,10pt]{jsarticle}
\usepackage{mylatex}
\usepackage{listings,jlisting}

\title{プロジェクト実習P PHP演習}
\author{C0116023 飯島貴政}
\date{2012年4月27日(金)}

\begin{document}
\maketitle
\lstset{
    %枠外に行った時の自動改行
    breaklines = true,
    %自動改行後のインデント量(デフォルトでは20[pt])
    breakindent = 10pt,
    %標準の書体
    basicstyle = \ttfamily\scriptsize,
    %関数名等の色の設定
    classoffset = 0,
    %枠 “t”は上に線を記載, “T”は上に二重線を記載
    %他オプション:leftline,topline,bottomline,lines,single,shadowbox
    frame = TBrl,
    %frameまでの間隔(行番号とプログラムの間)
    framesep = 5pt,
    %行番号の位置
    numbers = left,
    %行番号の間隔
    stepnumber = 1,
    %行番号の書体
    numberstyle = \tiny,
    %タブの大きさ
    tabsize = 4,
    %キャプションの場所(“tb”ならば上下両方に記載)
    captionpos = t
}

\section{目的}

今回の課題の目的を述べる.


\section{課題1}

\subsection{問題}

文字列「Hello World」をブラウザ上に表示するプログラムphp1.phpを作成せよ。
以降の課題はHTMLにPHPプログラムを埋め込むことによって構成すること。HTMLは5に準拠すること。


\subsection{ソースコード}

\lstinputlisting[caption=PHP1.php,label=pg:k1-1s]{src/php1.php}

\pgref{pg:k1-1s}のソース15行目で {\tt Hello, }という文字列を出力している.
次の16行目では{\tt Bye-bye 応用プログラミング!}という文字列を出力してい
る.

\subsection{ファイル構成}


Webアプリケーションは複数ファイルをいくつかのディレクトリに分けて保存す
ることが多い.その構成を図表で書いておくと状況が整理しやすいのでオススメ.


\subsection{実行結果}


\subsection{考察}

\pgref{pg:k1-1r} より,本課題の最低限の仕様を満たすことができたと言える.
しかし,見栄えや再利用性を考えた場合,あんなことやこんなことが必要になる
と考えられる.

…みたいに,実行結果を踏まえて現状を改善するにはどうすればいいか等を書く
と格好がつく.

\section{課題2}

複数の課題が出た場合はセクションを分けて書くこと.


\end{document}
